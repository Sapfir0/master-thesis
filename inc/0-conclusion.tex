\anonsection{Заключение}

В рамках выполнения НИР на основе анализа программ были выделе-ны и проанализированы ведущие тренды:
1.	Электронные IoT устройства, а также софт к ним, в Российской Федера-ции существуют и разрабатываются, но в значительно меньшем мас-штабе, чем на западе. 
2.	В странах ЕС и США IoT устройства получили более широкое распро-странение, начиная от медицинских умных датчиков до огромных си-стем, которые автоматизируют множество процессов в национальных парках.
3.	Использование современных инструментов front-end и back-end разра-ботки позволят создать масштабируемую систему для контроля и управления разными объектами.
Проведен анализ зарубежной (M. Stusek, K. Zeman, P. Masek, J. Hosek, J. Selova, K. Kim) и отечественной (Н.В. Рогачева, А.П. Карловский, В.Л. Можгинский, Ю.И. Иванов, К.В. Колоколова, А.Я. Номерчук, В.В. Соловьев, В.В. Щадрина, Д.Ю. Щербак, О.В. Гурин, В.П. Замышляев, Л.Е. Попок, И.А. Васюткина, Т.Н. Филимоненкова, А.Д. Григорьев) научной литературы и информационных источников (Официальные сайты Angular и React).
Анализ ведущих трендов социально-экономического и технологическо-го развития показал, что проблема развития IoT в Российской Федерации яв-ляется актуальной и востребованной, так как ее решение улучшит уровень автоматизации различных процессов.
Проведенный анализ зарубежных научных источников по теме иссле-дования «Исследование веб-технологий в стеке IoT» позволил выявить акту-альное состояние умных электронных устройств в Российской Федерации и в мире и варианты их развития. 
В ходе практики было полностью выполнено Индивидуальное задание.

\clearpage
