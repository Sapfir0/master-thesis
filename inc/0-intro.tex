\anonsection{Введение}

В настоящее время существует необходимость в автоматизации рутинных процессов. Более 70 лет мировой истории это представлялось людям, как прислуживающие роботы, но теперь эти процессы стали более известны как IoT.   Эта сфера помогает автоматизировать повторяющиеся задачи, проконтролировать протекание длительных процессов, уменьшить потребление ресурсов, таких как электроэнергия, и др.

В ряде западных стран такие системы интегрированы на многие предприятия, заводы, или просто вплетены в городскую среду, например, для контроля освещенности улиц. В Российской Федерации данная сфера находится на этапе активного внедрения у частных компаний, которые используют IoT для современных технологических производств, экономя средства на излишней работе станков. Внедрение IoT в Российской Федерации на данный момент не является достаточным, что делает актуальной тему данного исследования.

\textbf{Целью НИР} является анализ зарубежных и отечественных источников для формирования представления о веб-технологиях, используемых в сфере IoT.

\textbf{Объектом исследования} являются информационные системы, осуществляющие управления элементами IoT.

\textbf{Предметом исследования} являются результаты анализа отечественной и зарубежной литературы по теме исследования.

\textbf{Задачи исследования:}


\begin{itemize}
  \item Составить обзор российских и зарубежных научных исследований, посвященных IoT.
  \item Провести анализ отечественных научных исследований о существующих разработках в сфере IoT.
  \item Провести анализ зарубежных научных исследований в области проектирования IoT сервисов.
\end{itemize}


\textbf{Теоретические основы исследования.}	При анализе источников было выявлено, что тема внедрения элементов IoT освещена подробно, однако массового внедрения этой технологии на территории Российской Федерации не было.

В работе (Г.Ю. Портянкин, О.Ю. Рязанов) описаны принципы работы новой системы, которая позволяет решать ряд расчетных задач сельского хозяйства, получая и обрабатывая данные с умных устройств. По исследованию (Н.В. Рогачева) автор подытоживает текущие проблемы сферы и задачи, которые необходимо решить, чтобы продвинуть эту сферу вперед в Российской Федерации.

В трудах зарубежных авторов (M. Stusek, K. Zeman, P. Masek, J. Hosek, J. Sedova) подробно рассмотрены IOT протоколы передачи данных и перспективы их развития.

В рассмотренных источниках были приведены примеры реализации системы на IoT устройствах, однако сегодня сервисы, по которым были приведены примеры, не существуют по причине закрытия или нерелевантного принципа работы. Приведенные факторы обуславливают \textbf{актуальность темы научно-исследовательской работы}.

\textbf{Информационная база исследования:} eLIBRARY, CyberLeninka ResearchGate.

\textbf{Методы исследования.} Для решения задач исследования использовался комплекс теоретических и эмпирических взаимодополняющих методов исследования, среди которых ведущими были следующие методы: анализ, сравнение и обобщение, методы индукции и дедукции.

\textbf{Результаты исследования.} Проведенный анализ:

\begin{itemize}
  \item ведущих трендов социально-экономического и технологического развития показал, что проблема освещенности сферы интернета вещей в Российской Федерации является актуальной и востребованной, так как ее решение обеспечит создание лучших инструментов по управлению умными устройствами.
  \item отечественных и зарубежных источников позволил выделить и описать основные направления реализации системы с умными IoT датчиками, а также были инструменты, которые будут задействованы при ее проектировании.
\end{itemize}

\clearpage
