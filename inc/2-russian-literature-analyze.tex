\section{АНАЛИЗ ОТЕЧЕСТВЕННЫХ НАУЧНЫХ ИСТОЧНИКОВ ПО ТЕМЕ ИССЛЕДОВАНИЯ}


\subsection{Современное состояние IoT сферы}

В 1999 году одним из исследователей RFID-технологий Кевином Эштоном, возможно, впервые было употреблено словосочетание «Интернет вещей» (Internet of Things, IoT). Эштон использовал новоизобретенный тер-мин в ходе своей презентации для Procter\&Gamble, посвященной влиянию RFID на разные рынки.

В работе «Интернет вещей: обзор основных проблем и задач» (Н.В. Рогачева) поднимаются проблемы текущего состояния сферы [4].

\begin{itemize}
    \item Безопасность.
    \item Большое количество разнообразных несовместимых стандартов.
    \item Энергоэффективность.
    \item Повышенное потребление интернет-трафика.
    \item Недостаток специалистов с достаточной квалификацией в распределенных системах.
\end{itemize}


В России активно пытаются решить проблему
несовместимых стандар-тов и создают новые ГОСТы для стандартов IoT, такие как: 

\begin{itemize}
    \item ТК-194 «Киберфизические системы»,
    благодаря которому мы получи-ли в этом году серию национальных стандартов.
    \item ТК-26 «Криптографическая защита
    информации», где идут работы по улучшение безопасности протоколов Интернета вещей.
\end{itemize}

Также можно видеть улучшение взаимодействия между техническими комитетами, что положительно влияет на качество и сроки появления новых стандартов в России [5]. Стоит упомянуть компанию «Доверенная платфор-ма», которая сделала значительный вклад на этом поприще. Авторы статьи «Модуль IoT сервер ИКС “Агроаналитика IoT”» (Г.Ю. Портянкин, О.Ю. Рязанов) пишут о новой созданной ими системе, которая позволяет получать данные с устройств, хранить и обрабатывать полученные данные. Программа позволяет решать ряд расчетных задач сельского хозяй-ства, такие как подсчет топлива, стоянок, остановок и простоев, составление
логистической цепочки по урожаю, выявление неоднородности динамики биомассы, расчет площади обработки [6]. В работе
«Классификация систем и устройств IoT» (А.П. Карловский, В.Л. Можгинский) авторы разделяют IoT устройства на несколько
категорий [7]:

\begin{itemize}
    \item Сенсоры и исполнительные механизмы.
    \item Управление, локальная обработка данных и хранение информации.
    \item Канал связи со шлюзом в локальную сеть или Интернет. 
    \item Инфраструктура передачи данных, облачные технологии. 
    \item Программно-аппаратные сервисы интеграции устройств, управле-ния, визуализации и обработки информации.
    \item Аналитические системы.
\end{itemize}


\subsection{Варианты реализации системы IoT}

Существует обобщённая архитектура IoT систем, как написали авторы работы «Современное состояние дел в области создания систем с интеллекту-альными датчиками» Ю.И. Иванов, и др. [8]. Это embedded часть, которая находится непосредственно на устройстве, серверная часть, где хранятся и агреггируются все данные и front-end часть.  Общение может происходить по различным протоколам, таким как Zigbee, Wi-Fi, и др.
Для анализа front-end части сервиса была изучена работа «Сравни-тельный анализ JS фреймворков: Vue, React и Angular» (К.Н. Калугина), в которой автор подробно описал 3 популярных сегодня JavaScript фреймвор-ков. Автором были выделены фреймворки Angular и React, которые предла-гают высокую относительно Vue производительность и более широкую под-держку комьюнити. Также было отмечено, что у Vue значительно ниже порог входа, из-за которого он может быть предпочтителен для некоторых проек-тов. [9].
Для проектирования серверной части необходимо определиться с су-ществующими решениями для построения API. Ознакомившись с работами «Создание REST API микросервиса с использованием Flask» [10], «Разра-ботка серверной части web-приложений на JAVA» [11], «Архитектурные особенности проектирования и разработки веб приложений», [12], а также «Использование Node.JS для серверной архитектуры Web-приложения» [13] был сделан вывод, что разработанный сервис на фреймворке Flask обладает наибольшими конкурентными преимуществами, такими как поддерживае-мость, высокая скорость разработки, читаемость кода.


\clearpage

