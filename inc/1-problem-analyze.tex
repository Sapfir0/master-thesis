
\section{СОВРЕМЕННОЕ СОСТОЯНИЕ ПРОБЛЕМЫ}

В настоящее время, особенно в период пандемии COVID-19, возникла необходимость в обустройстве своего жилища, и развития сферы умных га-джетов. В мире растет количество «подключенных» устройств (по оценкам отраслевых аналитиков, их количество достигнет 20–50 млрд единиц к 2020 г.) и вместе с ним – количество примеров применения Интернета вещей (Internet of Things, IoT) в экономике: энергетике, промышленности, жилищно-коммунальном хозяйстве, сельском хозяйстве, транспорте, здравоохранении и др \cite{pwc_iot_perspectives}.

Как пишет Forbes, в 2022 году в IoT можно выделать пять мировых трендов развития [2]:

\begin{itemize}
  \item IoT в медицине – от фитнес браслетов до датчиков в реанимационных палатах.
  \item Защита передачи данных по сети.
  \item Edge computing – выполнение вычислений непосредственно на устрой-стве, без передачи на сервер, усиление мощностей каждого из устройств.
  \item Увеличение роли IoT в бизнесе, развитие VR, AR устройств, создание цифровых двойников объектов и даже предприятий целиком.
  \item Внедрение IoT датчиков в больших компаниях для экономии электро-энергии, интернет-трафика, или других потребляемых ресурсов.
\end{itemize}

В то же время, эксперты ЦНИИ «Электроника» совместно с АНО «Цифровая экономика» утверждают, что большинство производителей обо-рудования для IoT сферы в России используют отечественные комплектую-щие. У четверти из них доля российских деталей в цене продукции превыша-ет половину себестоимости.

Генеральный директор ЦНИИ «Электроника» высказывает обеспоко-енность тем, что индустрия в России находится на начальных этапах разви-тия, но вместе с этим примечает, что «Интернет вещей – это безусловный тренд», и с каждым годом растет количество компаний-производителей ум-ных устройств, а также компаний, использующих эти устройства для улуч-шения собственных экономических показателей. Также она утверждает, что данные исследования подтверждают высокую вакантность потребительской ниши [3].


\clearpage


