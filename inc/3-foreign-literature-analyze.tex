\section{АНАЛИЗ ЗАРУБЕЖНЫХ НАУЧНЫХ ИСТОЧНИКОВ ПО ТЕМЕ ИССЛЕДОВАНИЯ}


В работе «Water preservation using IoT: A proposed IoT system for de-tecting water pipeline leakage» (A. Abusukhon, F. Altamimi) авторы пишут о использовании IoT датчиков для детектирования протечек в водопроводе и автоматическом уведомлении всех подписанных пользователей для быстрого устранения проблемы сантехником [14]. 

Как пишут в казахстанской работе «Protecting sanctuaries and national parks in Kazakhstan with IoT technologies» (K. Kim), использование IoT устройств релевантно для интеграции в большие национальные парки, кото-рыми сложно или практически невозможно управлять с помощью неавтома-тизированных средств [15].

Также можно найти работы по созданию умных кампусов университе-тов, например, «IoT smart campus review and implementation of IoT applica-tions into education process of university» (A. Zhamanov, R. Suliyev, Z. Kaldyk-ulova, Z. Sakhiyeva. В этом тексте авторы говорят о интеграции умных устройств в университетах, чтобы улучшить процесс обучения студентов. Они интегрировали различные сенсоры на территории кампусов, а также со-здали облако для получения данных с них [16, 17]. 

Подтверждение этому можно найти и в текстах западных коллег, таким как - «IoT manager: An Open-Source IoT framework for smart cities» (L. Calde-roni, A. Magnani, D. Maio), в которой говорится о новом фреймворке для упрощения работы с большим (1000+) количеством подключенных устройств [18]. 

Этим использование устройств не ограничивается, и в работе «Fog computing and IoT for remote blood monitoring» (M. Orda-Zhigulina, D. Orda-Zhigulina) авторы рассказывают о возможностях применения интернета ве-щей в медицине, например, для создания интеллектуального непрерывного мониторинга глюкозы, который имеет возможность отправлять полученное информации о состоянии крови, отправлять на сервер, чтобы потом пользо-ватель мог получить полную информацию о состоянии своей крови [19].

Для изучения необходимых фреймворков для разработки были изуче-ны официальные ресурсы по React.JS, Angular. Преимущества React.JS были выделены такие 	[20]:

\begin{itemize}
    \item Virtual DOM повышает производительность высоконагруженных приложений, снижая вероятность перерисовки и улучшая пользо-вательский опыт.
    \item Использование изоморфного подхода позволяет производить рендеринг страниц быстрее.
    \item Повышенное переиспользование кода, можно использовать об-щий код для мобильного приложения и Web версии.
    \item Декларативное представление компонентов делает код более чи-таемым и предсказуемым.
\end{itemize}
  
У Angular они несколько отличаются [21]:

\begin{itemize}
    \item Angular – это фреймворк, который предлагает свою архитектуру построения системы, в отличии от React, который предоставляет только некоторые табличные методы.
    \item Наличие CLI системы для генерации стандартных решений.
    \item Строгая типизация кода по умолчанию.
    \item Наличие Dependency Injection из коробки.
\end{itemize}
Минусы Angular:
\begin{itemize}
    \item Посредственная документация.
    \item Большой объем результирующего кода.
    \item Высокий порог входа.
\end{itemize}


\clearpage


